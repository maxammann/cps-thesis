% Encoding: UTF-8

\documentclass[final,bibliography=totocnumbered]{include/sikseminar}

%%%%%%%%%%%%%%%%%%%%%%%%%%%%%%%%%%%%%%%%%%%%%%%%%%%%%%%%%%%%%%%%%%%%%%%%%%%%%%%%
%% Packages
%% Folgende Pakete werden bereits durch die sikseminar-Klasse geladen:
%\usepackage[german]{babel}
%\usepackage[headsepline]{scrlayer-scrpage}
%\usepackage{german}
%\usepackage{graphicx}
%\usepackage{textcomp}
%\usepackage{bibgerm}

\usepackage[utf8]{inputenc}
\usepackage{todonotes}
\usepackage{hyperref}
\hypersetup{colorlinks = true,linkcolor = black,citecolor = black,urlcolor = black,filecolor = black}

\graphicspath{{./figures/}}


\clubpenalty=10000
\widowpenalty=10000


\overfullrule=1mm
\begin{document}

% Thema
%\Title[Seminarvorlage]  % Kurztitel für Kopfzeilen
%      {Seminarvorlage mit Hinweisen zur Erstellung} % Ausführlicher Titel, der etwas mehr Platz braucht
\Title[Kurztitel für Kopfzeilen]{Ausführlicher Titel, der etwas mehr Platz braucht}
\makeTitle

\Author{Maximilian Ammann}
\Studiengang{Bachelor Informatik}
\makeAuthor

% Abgabedatum
\date{Datum des Vortrags}

\subject{Seminar Cyber-Physical Systems}

\maketitle

\begin{abstract}
\section*{Kurzfassung}
Eine kurze Zusammenfassung der Ausarbeitung mit 10-12 Zeilen Text.
\end{abstract}
\thispagestyle{empty}

\newpage

\tableofcontents

\newpage

\section{Section}\label{sec:section}

Test
dfd

\nocite{*}
\printbibliography
\todo{URL? ISBN? Bücher noch unklar welche Seiten.}

\listoftodos

\end{document}
