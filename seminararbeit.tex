% Encoding: UTF-8

\documentclass[final,bibliography=totocnumbered]{include/sikseminar}

%%%%%%%%%%%%%%%%%%%%%%%%%%%%%%%%%%%%%%%%%%%%%%%%%%%%%%%%%%%%%%%%%%%%%%%%%%%%%%%%
%% Packages
%% Folgende Pakete werden bereits durch die sikseminar-Klasse geladen:
%\usepackage[german]{babel}
%\usepackage[headsepline]{scrlayer-scrpage}
%\usepackage{german}
%\usepackage{graphicx}
%\usepackage{textcomp}
%\usepackage{bibgerm}

\usepackage[utf8]{inputenc}
\usepackage{todonotes}
\usepackage{hyperref}
\usepackage[T1]{fontenc}
\hypersetup{colorlinks = true,linkcolor = black,citecolor = black,urlcolor = black,filecolor = black}

\graphicspath{{./figures/}}


\clubpenalty=10000
\widowpenalty=10000


\overfullrule=1mm
\begin{document}

% Thema
%\Title[Seminarvorlage]  % Kurztitel für Kopfzeilen
%      {Seminarvorlage mit Hinweisen zur Erstellung} % Ausführlicher Titel, der etwas mehr Platz braucht
\Title[Security Considerations for CPS]{Security Considerations for Cyber-Physical Systems}
\makeTitle

\Author{Maximilian Ammann}
\Studiengang{Bachelor Informatik}
\makeAuthor

% Abgabedatum
\date{Datum des Vortrags \todo}

\subject{Seminar Cyber-Physical Systems}

\maketitle

\begin{abstract}
\section*{Kurzfassung}
Eine kurze Zusammenfassung der Ausarbeitung mit 10-12 Zeilen Text.
\end{abstract}
\thispagestyle{empty}
\newpage
\tableofcontents
\newpage

\section{Einführung}\label{sec:intro}
% Betriebsblindheit (Gollman)
Betriebsblindheit beschreibt einen Zustand in dem man sich Aufgrund weitreichendes Prozesswissens nicht über Fehler oder Mängel bewusst ist~\cite[S.~202]{GK16}.
Ist dieser Zustand einmal erreicht ist es schwer ihm wieder zu entfliehen und die Probleme in beispielsweise der Sicherheit\footnote{Sicherheit wird hier (falls nicht anders hingewiesen) unter dem Begriff der Security verstanden} zu erkennen.
Deshalb ist ein Ziel der Arbeit zu Motivieren sich mit Sicherheit auseinander zu setzen.
Besonders in Cyber-Physical Systems ist es wichtig sich mit Sicherheit zu befassen, da diese eine erhöhte Angriffsfläche für Angriffe bereitstellen, um diese Blindheit nicht auftreten zu lassen.
Gleichzeitig sind die Herausforderungen die ein solches System lösen soll und die damit verbundenen Anforderungen anderer Art als bei klassischen Systemen.

% Anforderungen an CPS: Predictability (Lee08)
% Challanges (SGL+08)
% Vision von CPS (RLS+10)
% IoT Constaint: Battery (YWY+17)
Bei Cyber-Physical Systems sind oft eingebettete echtzeit Systeme, welche eine hohe Verfügbarkeit, Robustheit, Widerstandsfähigkeit und Berechenbarkeit aufweisen müssen.
Die physische Welt macht Verfügbarkeit allerdings oft schwierig und ist alles andere als berechenbar~\cite{Lee08,SGL+08};.
Einsatzorte für diese Systeme könnten intelligente Stromnetze, symbiotische Sensornetzwerke für die Agrarwirtschaft und Katastrophenabwehr, medizinische oder assistierende Geräte, intelligente Verkehrssteuerung und intelligente Gebäude sein~\cite{RLS+10}.
In all diesen Beispielen spielt zudem ein hohes Maß an Vertrauen eine Rolle~\cite{SGL+08}.
Zudem unterliegt man konzeptionell bedingt auch einigen Restriktionen wie beispielsweise die die Bindung an eine Batterie oder leichte Bauweise~\cite{YWY+17}.
Es existiert also ein Unterschied in den Anforderungen an Cyber-Physical Systems zu klassischen Systemen, wie Anwendungsservern oder Heimcomputern.

In Kapitel~\ref{subsec:definition} wird zunächst der Begriff Sicherheit speziell für Cyber-Physical Systems definiert.
In den Kapiteln~\ref{subsec:angriffsziel},~\ref{subsec:angriffszenarien},~\ref{subsec:angreifergruppen} werden mögliche Angriffsszenarien beleuchtet um im Kapitel~\ref{subsec:gegenmassnahmen} adequate Gegenmaßnahmen darzustellen.
Zuletzt soll im Kapitel~\ref{sec:evaluation} auf vergangene Vorfälle im Bereich der Security eingegangen werden.





% Stuxnet (Langer)
% Definition (BG11)


\section{Aspekte eines Angriffes}\label{sec:aspekte}

% Clickbait bei New im Bereich Security

\subsection{Security Definition f\"ur Cyber-Physical Systems}\label{subsec:definition}
% computer, information, network, communication, physical security
% CIA triad (Fink) (WYX+10)
% Erweiterung von CIA: Veracity, Plausability (Gollman)
% Fingerprinting, Network Separation, End System Security (Gollman)
% Erweiterung von CIA: Authenticity (WYX+10, II. A.)

% CPS-Security = Cyber und Physical Security

\subsection{Cyber-Physical Systems als Angriffsziel}\label{subsec:angriffsziel}
% (Gollman, 1)
% Survey: medial, IoT, smart grid, power plant (Humayed)
% IoT (FPA+18) (YWY+17)
% Security Considerations in Cloud (SPB+16)

\subsection{Angriffszenarien}\label{subsec:angriffszenarien}
% Sabotage und Spionage
% Punkte wo angegriffen werden kann (Fink, Figure 1.1) (Cardenas 2008, Figure 3)
% Migration von Legacy Systemen ist kritisch (Gollman, 1.)
\subsubsection{Klassische Szenarien}
% Unterschied zu klassischen Systemen (Cardenas 2008, 3.)
% IoT Unterschiede zu klassischem (FPA+18)
% Workflow von CPS (WYX+10, II.)
\subsubsection{Denial-of-Service Angriff}
% (WYX+10, II. B.)
% (Cardenas 2008, 2.1)
\subsubsection{Man-in-the-Middle Angriff}
% (WYX+10, II. B.)
\subsubsection{T\"auschungsangriff (Deception)} % Deception
% (Cardenas 2008, 2.1)
\subsubsection{Lauschangriff (Eavesdropping)} % Eavesdropping
% (WYX+10, II. B.)
\subsubsection{Compromised-Key Angriff}
% (WYX+10, II. B.)

\subsection{Angreifergruppen}\label{subsec:angreifergruppen}
% Cyber-Kriminelle
% Verärgerte Mitarbeiter/Private Gründe
% Terroristen, Aktivisten, kriminelle Gruppen
% Staaten (Cardenas 2009, 2.) (WYX+10, II. C.)


\subsection{Gegenmaßnahmen}\label{subsec:gegenmassnahmen}
% Properties: Safety, Security, Reliability, Resilence (LLZ+14)

% Least Privilege (Gollman)
% Need-to-Know (Gollman)
% Separation (Gollman)

% Proactive Mechanisms, Reactive Mechanisms, Design and Analysis Principles (Cardenas 2008)

% Context Aware Security: Sensing, Cyber, Control, Physical Security (WYX+10)

\subsubsection{Modellierung von Cyber-Physical Systems (als Grundlage)}
% Argument: Neue Grundlage für Embedded Systems/CPS: Modelbasiert anstatt Programm (Lee08)
Noch nicht sicher.

Um hierbei dem Standard an Sicherheit zu genügen schlägt \citeauthor{Lee08} sogar vor eine neue Grundlage für diese Systeme zu etablieren~\cite{Lee08}.

\subsubsection{Physische Maßnahmen}
% (Cardenas)

\subsubsection{Organisatorische Maßnahmen}
% Transdiszipliär (Brazell14)

\subsubsection{Präventive Maßnahmen}
% (Cardenas 2009, 4.)

\subsubsection{Detektion und Wiederherstellung (Detection and Recovery)}
% (Cardenas 2009, 4.)

\subsubsection{Widerstandsfähigkeit (Resilience)} % Resilience
% (Cardenas 2009, 4.)

% End-to-End Security (Fink)

\subsubsection{Abschreckung (Deterrence)} % Deterrence
% (Cardenas 2009, 4.)

\section{Evaluation aktueller Maßnahmen}\label{sec:evaluation}

\newpage
\nocite{*}
\printbibliography
\newpage
% \listoftodos
\end{document}
