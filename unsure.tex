\subsection{\glsentrytext{cps} als Angriffsziel}\label{subsec:angriffsziel}
\todo{Noch sehr unzufrieden mit diesem Kapitel, Eventuell anderer Aufbau: Problem -> Beispiel}
%Es gibt eine Vielzahl von Anwendungen für \cps, wie beispielsweise industrielle Kontrollsysteme, intelligente Stromnetze, medizinische Geräte und intelligente Autos~\cite{HLL+17}.
\todo{Acronym ICS, ICCP, DNP3, Modbus}
Hierbei sind die unterschiedlichsten Protokolle und Busse zur Kommunikation.
ICCP, Modbus oder auch DNP3, die Verwendung bei Industrial Control Systems und intelligenten Stromnetze finden~\cite{HLL+17}, implementieren keine Verschlüsselung oder Authentifizierung \cite{HUM 124,HUM 38,HUM 69}, wodurch Lauschangriffe (siehe Kapitel~\ref{subsec:lauschen}) möglich sind.
Es ist also wichtig offene Protokolle, wie ICCP oder TCP/IP, zu verwenden, sodass man Grundlegende Eigenschaften der Confidentiality und Authenticity bei einem \cps bestimmen kann und falls nötig diese nachrüsten kann.

Bei medizinischen Geräten ist eine kabellose Verbindung oft notwendig~\cite{HLL+17}, da Aktualisierungen über ein Kabel nur schwer umzusetzen sind.
Besonders in diesem Bereich sollte das Bewusstsein für Sicherheit wachsen, da direkt Menschenleben und deren Privatsphäre, davon abhängen wie sicher diese Systeme sind.

Durch diese Vielfalt an Kommunikationsmöglichkeiten und der Heterogenität der einzelnen Komponenten gibt es auch eine dementsprechend große Angriffsfläche~\cite{HLL+17}.
Zudem wird bei der Entwicklung oft versucht durch proprietäre Protokolle und deren Obskurität Sicherheit zu gewährleiten~\cite{HLL+17,SJT2008}\todo{cite NIST}.

% Die Umgebung des \cps ist meist vor der Entwicklung noch nicht bekannt oder wird ignoriert~\cite{CAS08,Ericsson2010}.

% IoT (FPA+18) (YWY+17)
% Security Considerations in Cloud (SPB+16)

% Survey: medial, IoT, smart grid, power plant (Humayed)

