% Encoding: UTF-8
\documentclass[final,bibliography=totocnumbered]{include/sikseminar}

%% Folgende Pakete werden bereits durch die sikseminar-Klasse geladen:
%\usepackage[german]{babel}
%\usepackage[headsepline]{scrlayer-scrpage}
%\usepackage{german}
%\usepackage{graphicx}
%\usepackage{textcomp}
%\usepackage{bibgerm}

\usepackage[utf8]{inputenc}
\usepackage{todonotes}
\usepackage{hyperref}
\usepackage[T1]{fontenc}
\hypersetup{colorlinks = true,linkcolor = black,citecolor = black,urlcolor = black,filecolor = black}
\usepackage{enumitem}
\usepackage[acronym,toc,section=section,numberedsection]{glossaries}
\glstoctrue
\makeglossaries
\newacronym[shortplural={CPS},longplural={Cyber-physical Systems}]{cps}{CPS}{Cyber-physical System}


\usepackage[autostyle=true,german=quotes]{csquotes}
%\usepackage{natbib}
\graphicspath{{./figure/}}


\clubpenalty=10000
\widowpenalty=10000


\overfullrule=1mm

\newcommand{\fb}[1]{\dofb#1}
\newcommand{\dofb}[1]{\textbf{#1}\nobreak\hspace{0pt}}


\begin{document}

\Title[Security Considerations for \glsentrytext{cps}]{Security Considerations for Cyber-Physical Systems}
\makeTitle

\Author{Maximilian Ammann}
\Studiengang{Bachelor Informatik}
\makeAuthor
\date{Datum des Vortrags \todo}
\subject{Seminar Cyber-Physical Systems}

\maketitle

\begin{abstract}
\section*{Kurzfassung}
Eine kurze Zusammenfassung der Ausarbeitung mit 10-12 Zeilen Text.
\end{abstract}
\thispagestyle{empty}
\newpage
\tableofcontents
\newpage

\section{Einführung}\label{sec:intro}
% Anforderungen an CPS: Predictability (Lee08)
% Challanges (SGL+08)
% Vision von CPS (RLS+10)
% IoT Constaint: Battery (YWY+17)
\glspl{cps} sind meist eingebettete echtzeit Systeme, welche eine hohe Verfügbarkeit, Robustheit,Widerstandsfähigkeit und Berechenbarkeit aufweisen müssen.
Die physische Welt macht hohe Verfügbarkeit allerdings oft schwierig da sie alles andere als berechenbar ist~\cite{Lee08,SGL+08}.
Einsatzorte für diese Systeme könnten intelligente Stromnetze, symbiotische Sensornetzwerke für die Agrarwirtschaft und Katastrophenabwehr, medizinische oder assistierende Geräte, intelligente Verkehrssteuerung und intelligente Gebäude sein~\cite{RLS+10}.

In all diesen Beispielen spielt ein außerordentliches Maß an Vertrauen eine Rolle~\cite{SGL+08}.
Dieses ist zwar auch bei klassischen Systemen gefordert, allerdings nicht in gleicher Weise, da sie nicht in gleicher Weise an physische Prozesse gekoppelt sind~\cite{BG11}.
Zudem unterliegt man konzeptionell bedingt auch einigen Restriktionen wie beispielsweise die Bindung an eine Batterie oder eine leichte Bauweise~\cite{YWY+17}.
Es existiert also ein Unterschied in den Anforderungen an \glspl{cps} und klassischen Systemen, wie Anwendungsservern oder Heimcomputern, sodass diese beiden in Bezug auf Sicherheit anders betrachtet werden müssen.

In dem Kapitel~\ref{sec:bedeutung-sicherheit} wird zunächst das Zusammenspiel von ''cyber'' und ''physical'' im Bezug auch Sicherheit geklärt.
Zudem werden mögliche Ziele und Angreifergruppen in den Kapitel~\ref{subsec:angriffsziel}~und~\ref{subsec:angreifergruppen} beleuchtet.
Im Kapitel~\ref{sec:angriffszenarien} werden mögliche Szenarien und im Kapitel ~\ref{subsec:angreifergruppen} dazu Gegenmaßnahmen dargestellt.
Zuletzt soll im Kapitel~\ref{sec:diskussion} diskutiert werden ob die genannten Gegenmaßnahmen  für die Szenarien ein adäquate Lösung darstellen.
% Stuxnet (Langer)


\section{Bedeutung von Sicherheit in \glsentrytext{cps}}\label{sec:bedeutung-sicherheit}
Die Bedeutung von Sicherheit in ''cyber'' Systemen und ''physical'' Systemen unterscheiden sich intrinsisch.
Um Klarheit zu schaffen wo diese Unterschiede liegen soll Sicherheit für diese beiden zunächst definiert werden, sodass anschließend Sicherheit für \gls{cps} definiert werden kann.

\subsection{Definition für Sicherheit von \glsentrytext{cps}}\label{subsec:definition}
Unter Cybersicherheit versteht man im Allgemein Informationssicherheit.
Diese Sicherheit kann man durch drei Grundlegende Prinzipien, dem sog CIA-triad, definieren~\cite[,S.~2]{Cherdantseva2013,SFJ17a}:
\begin{itemize}[noitemsep,nolistsep]
\item \fb{Confidiality} - Nur autorisierte Teilnehmer können auf die Infrastruktur zugreifen.
\item \fb{Integrity} - Die Infrastruktur kann nur von autorisierten Teilnehmer verändert werden.
\item \fb{Availability} - Die Infrastruktur ist für autorisierte Teilnehmer angemessen verfügbar.
\end{itemize}
Zwischen diesen drei Prinzipien muss bei der Entwicklung ein sinnvolles Gleichgewicht gefunden werden.
\citeauthor{GK16} beschreiben, dass bei klassischen Cybersystemen der Fokus auf \fb{Confidiality}, bei \glspl{cps} bisher allerdings eher auf \fb{Availability} liegt.
Sie schlagen außerdem vor die Sicherheitsdefinition für \glspl{cps} durch die Prinzipien \fb{Veracity} und \fb{Plausibility} zu ergänzen.

% Definition (BG11)
% computer, information, network, communication, physical security
% CIA triad (Fink) (WYX+10)
% Erweiterung von CIA: Veracity, Plausability (Gollman)
% Fingerprinting, Network Separation, End System Security (Gollman)
% Erweiterung von CIA: Authenticity (WYX+10, II. A.)

% CPS-Security = Cyber und Physical Security

\subsection{Angreifergruppen}\label{subsec:angreifergruppen}
% Cyber-Kriminelle
% Verärgerte Mitarbeiter/Private Gründe
% Terroristen, Aktivisten, kriminelle Gruppen
% Staaten (Cardenas 2009, 2.) (WYX+10, II. C.)

\subsection{\glsentrytext{cps} als Angriffsziel}\label{subsec:angriffsziel}
% (Gollman, 1)
% Survey: medial, IoT, smart grid, power plant (Humayed)
% IoT (FPA+18) (YWY+17)
% Security Considerations in Cloud (SPB+16)

\section{Angriffszenarien}\label{sec:angriffszenarien}
% Sabotage und Spionage
% Punkte wo angegriffen werden kann (Fink, Figure 1.1) (Cardenas 2008, Figure 3)
% Migration von Legacy Systemen ist kritisch (Gollman, 1.)
\subsection{Unterschiede zu klassische Szenarien}\label{subsec:klassisch}
% Unterschied zu klassischen Systemen (Cardenas 2008, 3.)
% IoT Unterschiede zu klassischem (FPA+18)
% Workflow von CPS (WYX+10, II.)
\subsection{Denial-of-Service Angriff}\label{subsec:dos}
% (WYX+10, II. B.)
% (Cardenas 2008, 2.1)
\subsection{Man-in-the-Middle Angriff}\label{subsec:mitm}
% (WYX+10, II. B.)
\subsection{T\"auschungsangriff (Deception)}\label{subsec:tauschung} % Deception
% (Cardenas 2008, 2.1)
\subsection{Lauschangriff (Eavesdropping)}\label{subsec:lauschen} % Eavesdropping
% (WYX+10, II. B.)
\subsection{Compromised-Key Angriff}\label{subsec:key}
% (WYX+10, II. B.)


\section{Gegenmaßnahmen}\label{sec:gegenmassnahmen}
% Properties: Safety, Security, Reliability, Resilence (LLZ+14)

% Least Privilege (Gollman)
% Need-to-Know (Gollman)
% Separation (Gollman)

% Proactive Mechanisms, Reactive Mechanisms, Design and Analysis Principles (Cardenas 2008)

% Context Aware Security: Sensing, Cyber, Control, Physical Security (WYX+10)

\subsection{Modellierung von \glsentrytext{cps}}\label{subsec:modellierung}
% Argument: Neue Grundlage für Embedded Systems/CPS: Modelbasiert anstatt Programm (Lee08)
Noch nicht sicher.

Um hierbei dem Standard an Sicherheit zu genügen schlägt \citeauthor{Lee08} sogar vor eine neue Grundlage für diese Systeme zu etablieren~\cite{Lee08}.

\subsection{Physische Maßnahmen}\label{subsec:physisch}
% (Cardenas)

\subsection{Organisatorische Maßnahmen}\label{subsec:orga}
% Transdiszipliär (Brazell14)
% Betriebsblindheit (Gollman)

\subsection{Präventive Maßnahmen}\label{subsec:präventiv}
% Security by Obscurity (Scarfone2008)
% (Cardenas 2009, 4.)

\subsection{Detektion und Wiederherstellung (Detection and Recovery)}\label{subsec:detektion}
% (Cardenas 2009, 4.)

\subsection{Widerstandsfähigkeit (Resilience)}\label{subsec:widerstand} % Resilience
% (Cardenas 2009, 4.)

% End-to-End Security (Fink)

\subsection{Abschreckung (Deterrence)}\label{subsec:abschreckung} % Deterrence
% (Cardenas 2009, 4.)

\section{Sind die Gegenmaßnahmen für potenzielle Szenarien ausreichen?}\label{sec:diskussion}

Sind sie Notwendig?
Ja! -> Sind sie Ausreichend?

\newpage
\printglossary[type=\acronymtype]
~\nocite{*}

\printbibliography
\newpage
% \listoftodos
\end{document}
